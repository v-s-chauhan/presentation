% This text is proprietary.
% It's a part of presentation made by myself.
% It may not used commercial.
% The noncommercial use such as private and study is free
% June 2019
% Author: Vivek Kumar Singh
% ITPC Kokata
% www.bsnl.co.in/
% 
%
\documentclass{beamer}
\usepackage{beamerthemeshadow}
\begin{document}
\title{ Introduction to Shell Scripting } 
\author{Vivek Kumar Singh}
\date{\today}

\begin{frame}
\titlepage
\end{frame}

%\begin{frame}\frametitle{ Table of Contents }\tableofcontents
%\end{frame}

\begin{frame}\frametitle{ Agenda }

\begin{itemize}
\item Defining and Using Shell Scripts
\item Standard Input, Output and Error
\item Uing variables and Operators in Shell Scripts
\item How to use conditional statements and Loops
%\item How to use Loops for repetitive Execution
\item Using Functions to use a block of code whenever needed
%\item Ho to redirect Standard Output and Error
\item Examples and Use Cases
\item Summary
\end{itemize}

\end{frame}

\section{Introducton}
\begin{frame}\frametitle{ Definition }
\begin{block}{ Shell }
Shell is a commandline interpreter. It  translates commands entered by the user and converts them into a language that is understood by kernel.
\end{block}
\pause
\begin{block}{ Shell Script }
Shell Script has list of commands in the order of execution.
\end{block}
\end{frame}

\begin{frame}\frametitle{Creating and Executing Scripts}

\begin{block}{ create a file (vi test.sh) }
\#\!/bin/bash	\newline
echo "Hello Everyone"
\end{block}
\pause
\begin{block}{ make the fie executable }
chmod +x test.sh 
\end{block}
\pause
\begin{block}{ run the Script }
./test.sh
\end{block}

\end{frame}

\section{ Script Essentials}
\subsection{ Variables and Operators }
\begin{frame}\frametitle{ Variables }

\begin{block}{ Types Of Variables }
System Variables(Environment variables ) \newline
User Defined Variables \newline
Special Variables
\end{block}
\pause
\begin{block}{ Special Variables (scriptname arg0 arg1 arg2 arg3}
\$0, \$1, \$2 scriptname, first argument, second argument \newline
%\$1  first argumrnt \newline
%\$2  second arguent \newline
\$\# number of arguments \newline
\$?  Exit status of Last Cmmand
\end{block}
\pause
\begin{block}{ Scope of Variable }
Local and Global \newline
by default variables are global
\end{block}

\end{frame}
\begin{frame}\frametitle{ Variable Assignment and Use }

\begin{block}{ Variable Assignment }
var1=45 \newline
var2=vivek
\end{block}
\pause
\begin{block}{ Using Variables }
echo "\$1" \newline
echo "my name is \$var2"
\end{block}
\pause
\begin{block}{ Taking User input }
read name \newline
echo "my name is \$name"
\end{block}
 
\end{frame}
\begin{frame}\frametitle{ Basic Operators }

\begin{block}{ Types Of Operators }
Arithmetic Operators \newline
Relational Operators \newline
Boolean Operators \newline
String Operators  \newline
File Test Operators  \newline
\end{block}
\pause
\begin{block}{ Arithmetic Operators }
\#\!/bin/bash \newline
a=10 ; b=20 \newline
%b=20 \newline
echo \`{}expr \$a + \$b\`{} \newline
echo \`{}expr \$b / \$a\`{}
\end{block}

\end{frame}

\begin{frame}\frametitle{ Standard Input Input, Output and Error }

\begin{block}{ Standard Input }
Taking input from keyboard 
\end{block}
\pause

\begin{block}{ Standard Output }
Output descriptor Mainly on attached display
\end{block}
\pause
\begin{block}{ Standard Error }
Default error output device 
\end{block}

\end{frame}

\subsection{ Conditional Statements }
\begin{frame}\frametitle{ Conditional Statements }

\begin{block}{ if then else fi }
If evaluates a condition, If a condition is true then if block code is executed. On false condition, else block code is executed, but its optional
\end{block}
\pause
\begin{block}{ Case esac }
It allows us to execute different set of instructions against different values of a varaible
\end{block}
\pause
\begin{block}{ break and continue }
The break statement terminates the current loop and passes program control to the command that follows the terminated loop. \newline \pause
continue causes a jump to the next iteration of the loop, skipping all the remaining commands in that particular loop cycle
\end{block}

\end{frame}
\subsection{ Shell Loops }
\begin{frame}\frametitle{ Shell Loops }

\begin{block}{ For Loop }
executes commands for given set of values
\end{block}
\pause
\begin{block}{ While Loop }
executes commands as long as the given condition evaluates to true
\end{block}
\pause
\begin{block}{ Until Loop }
execute commands as long as the given condition evaluates to false
\end{block}
\pause
\begin{block}{ Infinite Loop }
A loop that runs forever
\end{block}

\end{frame}
\subsection{ Functions }
\begin{frame}\frametitle{ Functions }

\begin{block}{ Definition }
Functions are blocks of code which could be reused anywhere in the code.
\end{block}
\pause
\begin{exampleblock}{ Example: Functions and Global Variable }
\#!/bin/bsh \newline
username=vivek \newline
echo "Outside Function: $username" \newline
func() \newline
\{ \newline
echo "Inside Function: $username" \newline
\} \newline
ffunc
\end{exampleblock}

\end{frame}
\subsection{ Redirection }
\begin{frame}\frametitle{ Redirecting Output and Error }

\begin{block}{ append and overwrite type of redirects }
echo Hello \textgreater file1.txt   \# used to overwrite text to the file\newline
echo Hello \textgreater\textgreater file2.txt \# used to append text to the file
\end{block}
\pause
\begin{block}{ Redirecting Standard Output }
./scriptname.sh \textgreater logfile.txt
\end{block}
\pause
\begin{block}{ Redirecting Standard Error }
./scriptname.sh 2\textgreater logfile.txt
\end{block}
\pause
\begin{block}{ Redirecting Both Standard Output and Error }
./scriptname.sh \textgreater/tmp/out.txt 2\textgreater/tmp/error.log
\end{block}

\end{frame}

%%%%%%%%%%%%%%%%%%%%%%%%%%%%%%%%%%%%%%%%%%%%%%%%%%%%%%%%%%%%%%%%%%%%%%%%
%\begin{frame}\frametitle{unnumbered lists}
%\begin{itemize}
%\item Introduction to  \LaTeX 
%\item Course 2
%\item Termpapers and presentations with \LaTeX
%\item Beamer class
%\end{itemize}
%\end{frame}

%\begin{frame}\frametitle{lists with pause}
%\begin{itemize}
%\item Introduction to  \LaTeX \pause
%\item Course 2 \pause
%\item Termpapers and presentations with \LaTeX \pause
%\item Beamer class
%\end{itemize}
%\end{frame}

%\begin{frame}\frametitle{numbered lists with pause}
%\begin{enumerate}
%\item Introduction to  \LaTeX \pause
%\item Course 2 \pause
%\item Termpapers and presentations with \LaTeX \pause
%\item Beamer class
%\end{enumerate}
%\end{frame}
%%%%%%%%%%%%%%%%%%%%%%%%%%%%%%%%%%%%%%%%%%%%%%%%%%%%%%%%%%%%%%%
\section{Examples and Use Cases}
\subsection{Examples}

\begin{frame}\frametitle{ Example }
\begin{block}{Understanding Local Variables}
Create a local variable and verify that its scope is local to its loop only.
\end{block}
\pause
\begin{exampleblock}{ Solution Code }
\#\!/bin/bash \newline
username=vivek \newline
echo "Outside Function: \$username" \newline
func\(\) \{ \newline
local username=vsingh \newline
echo "Inside Function: \$username" \} \newline
func \newline
echo "Outside Function: \$username"
\end{exampleblock}
\end{frame}

%%\begin{frame}\frametitle{ Example }
%%\begin{block}{Title of the Example}
%%Probem Statement
%%\end{block}
%%\pause
%%\begin{exampleblock}{  Solution Code }
%%Solutin Script
%%\end{exampleblock}
%%\end{frame}

%%\begin{frame}\frametitle{ Example }
%%\begin{block}{Title of the Example}
%%Probem Statement
%%\end{block}
%%\pause
%%\begin{exampleblock}{  Solution Code }
%%Solutin Script
%%\end{exampleblock}
%%\end{frame}


\subsection{Use Cases}
\begin{frame}\frametitle{ Use Case : Matching User }

\begin{block}{ Greeting for current user }
Take username as input, If it matches with current user it should greet you, else it should say "Try Again"
\end{block}
\begin{exampleblock}{ Solution Code}
\#\!/bin/bash  \newline
echo "Please Enter username " ; read username \newline
if [ "\$username" == "\$USER" ] \newline
then  \newline
echo "Hello \$username" \newline
else \newline
echo "Try Again" \newline
fi

\end{exampleblock}


\end{frame}
\begin{frame}\frametitle{ Use Case : Host Discovery }

\begin{block}{ Host Discovery Alert}
If destination server is not reachable, sen an Email
\end{block}
\begin{exampleblock}{ Solution Code( Run script as ./chkhost.sh host1 host2 host3) }
\#\!bin/bash \newline
for i in \$@ \newline
do \newline
ping -c 1 \$i \textgreater/dev/null \newline
if [ \$? -ne 0 ] \newline
then \newline
echo "\$i is down" \textbar mail -s "Host Down" itpc.vivek@gmail.com \newline
fi \newline
done 
\end{exampleblock}
\end{frame}

\section{ Summary }
\subsection{ Summary }
\begin{frame}\frametitle{ Summary}

\begin{block}{ We Discussed }
How to write small scripts \newline
How to use variables and inputs \newline
How to use commandline arguments in script \newline
How to use control statements and Loops \newline
How to use functins \newline
How to use Redirecion
\end{block}
\pause
\begin{block}{ Scope }
Further scripts could be automated using crontab
\end{block}

\end{frame}


%\begin{frame}\frametitle{ Thanks and Suggestions }
\subsection{ Thanks }

\begin{frame}

\begin{block}{       Thank You  }
%thank you
\end{block}
\begin{block}{    Questions and Suggestions }
%           Questions and Suggestions
\end{block}


\end{frame}



\end{document}


